\documentclass{jsreport}

\title{釧路工業高等専門学校 サイバーセキュリティ同好会\\ \Large 同好会会則}
\author{サイバーセキュリティ同好会会長}

\begin{document}

\maketitle
\tableofcontents

%----------------------------------
\chapter{総則}
\section{名称}
本会は、釧路工業高等専門学校サイバーセキュリティ同好会と称する。

\section{所在地}
本会は、北海道釧路市大楽毛西2丁目32番1号 釧路工業高等専門学校に置く。

\section{目的}
本会は、CTFやリバースエンジニアリング等の活動を通じて、サイバーセキュリティに関する知識の向上を目指す。

\section{活動内容}\label{sec:about_of_activity}
本会は、前条目標を達成するために次の活動をする。 
\begin{enumerate}
    \item 毎週月曜日から金曜日、オンラインまたは釧路工業高等専門学校で活動を行う。ただし、定期試験2週間前および定期試験期間中は一切の活動を行わない。
    \item 随時、CTF大会への参加を行う。
    \item サイバーセキュリティに対する知見を深めるため、定期的に技術交流会を開く。
\end{enumerate}

%----------------------------------
\chapter{会員}
\section{資格}
以下にすべて該当するものが参加資格を有する。
\begin{enumerate}
    \item 釧路工業高等専門学校 創造工学科、または専攻科に所属する者。
    \item 本会の趣旨に賛同し、入会を希望する者。
    \item 他の部活動を兼部する場合は、その部活動より兼部の許可が降りている者。
\end{enumerate}

\section{会員条件}
会員は週に1度活動に参加する。ただし、第\ref{sec:about_of_activity}条に基づくものとする。

ただし、次の条件を満たすと判断した場合はこれを免除する。
\begin{enumerate}
    \item 釧路工業高等専門学校が定める休日・長期休暇・活動時間にあたる場合。
    \item 釧路工業高等専門学校が休業している場合。
    \item 体調不良等、正当な理由により活動に参加できない場合。
\end{enumerate}

\section{入会・退会}
\begin{enumerate}
    \item 第5条の条件を満たす者が、本会則に同意の上で執行部へ入会を申請し、執行部の承認を得るものとする。
    \item 会員から退会の申し出があった場合には、執行部の決議をもって退会したものとする。
    \item 本会則に違反した会員は、執行部の決議をもって会員資格を失う。
    \item 前項によって会員資格を剥奪されたものは執行部全員の承認を持って退会したものとする。
\end{enumerate}

%----------------------------------
\chapter{執行部}
\section{執行部}
本会に執行部を設け、下記の役員を設ける。
\begin{itemize}
    \item 会長 1名
    \item 副会長 1名
    \item 会計 1名
    \item 監査 1名
\end{itemize}

\section{執行部の選出}
執行部役員の任期は1年とし、再選は妨げられない。役員の選任は立候補制とし、総会で承認を得るものとする。

\section{執行部の権限}
執行部は、会員に対して第\ref{sec:about_of_activity}条に示された活動の一部を強制することができる。ただし、以下の条件に従う。
\begin{itemize}
    \item 通達は活動開始時期から一定期間を設け、必ず事前に行う。
    \item 本会の活動を活発化するためのものであり、会員に附下をかける目的ではないこと。
\end{itemize}
ただし、次の条件を満たすと判断された場合免除や一部変更を認める。
\begin{itemize}
    \item 体調不良等やむを得ない場合。
    \item その他執行部が判断した場合。
\end{itemize}
この他、執行部は以下を決定することができる。
\begin{itemize}
    \item 会員の入会、退会、および除籍。
    \item 会員資格の消失・回復。
\end{itemize}

\section{異議申し立て}
\begin{enumerate}
    \item 会員は執行部の判断、運営に対し異議を申し立てることができる。
    \item 執行部は異議申し立てを受けた場合、再度審議を行いその結果を通知する。
    \item 前項の手続きにおいても異議が解決しなかった場合、総会に判断を委ねることができる。
\end{enumerate}

\section{不信任決議}
\begin{enumerate}
    \item 会員は、現執行部が本会を運営することが不適当であると判断した場合、不信任決議案を発議することができる。
    \item 不信任決議案を発議した場合は、臨時総会を開くものとする。
    \item 総会にて不信任決議案が可決された場合、現執行部役員は退任し、即日選任を行う。
\end{enumerate}

%----------------------------------
\chapter{財務}
\section{会計の責務}
\begin{enumerate}
    \item 本会における会計業務は会計が行う。
    \item 会計は、執行部へ定期的な会計報告を行う。会員からの要求があった場合は、会員にも公表を行う。
    \item 物品を購入する場合は、原則本会が所有する資金から支出を行う。足りない場合は、執行部全員の承認を得て会員から一律金額を徴収できる。
    \item 資金については、会計が適切に管理する。
\end{enumerate}

\section{物品購入および会費}
\begin{enumerate}
    \item 物品購入は、執行部の決定、あるいは必要に応じて会員が会計に要求することで行う。購入は会計が行う。
    \item 会費の徴収については、執行部内で協議し、総会で決定される。
    \item 会費の徴収が決定した場合、会計が徴収を行う。
\end{enumerate}

%----------------------------------
\chapter{総会}
\section{成立}
総会は、全会員の3分の2以上の出席をもって成立とする。

\section{総会の開催}
\begin{enumerate}
    \item 定例総会は年に2度、5月と10月に行い、役員の選任や決算の承認等を行う。
    \item 会員が総会の開催を必要と判断した場合、臨時総会を行うことができる。
\end{enumerate}

\section{決議}
\begin{enumerate}
    \item 総会は出席者の過半数の同意をもって決定する。
    \item 総会で決定された内容は執行部の判断よりも優先される。
\end{enumerate}

\section{会則の期限}
\begin{enumerate}
    \item 本会則はその期限を1年とする。
    \item 前項の期限前日までに改正案が発議されなかった場合、自動的に期限日から起算して1年延長する。
\end{enumerate}

\section{会則の改正}
\begin{enumerate}
    \item 本規則の改正は会員の要望、その他執行部役員が必要であると認めた場合、これを発議し総会の承認を得なければならない。
    \item 前項の承認を得たときは直ちにこれを施行する。
\end{enumerate}

%----------------------------------
\chapter{附則}
\section{設立年月日}
同好会が承認された日を設立日とする。

\section{規約施行日}
同好会が承認された日を規約施行日とする。
\end{document}
